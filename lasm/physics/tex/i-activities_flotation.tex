\section{Flotation}

\subsection{Conditions of Floatation}

\subsubsection*{Learning Objectives}
\begin{itemize}
\item{To explain the conditions for a floating body.} 
\end{itemize}

\subsubsection*{Materials}
Water, fresh egg, salt, beaker*

\subsubsection*{Activity Procedure}
\begin{enumerate}
\item{Fill a beaker with water.} 
\item{Release a fresh egg on the surface of water slowly and observe its position.} 
\item{Add some salt while stirring and observe the position of the egg.} 
\item{Continue adding salt and observe the position of the egg.} 
\end{enumerate}

\subsubsection*{Results and Conclusion}
The egg sinks to the bottom of the container because its density is greater than that of water. After adding the salt, the egg rises and finally floats on the surface of water. This is because the density of the water becomes higher than that of the egg.  

\subsubsection*{Clean Up Procedure}
\begin{enumerate}
\item{Collect all the used materials, cleaning and storing items that will be used later.} 
\end{enumerate}

\subsubsection*{Discussion Questions}
\begin{enumerate}
\item{Why does the egg sinks?}
\item{What causes the egg to float?}
\end{enumerate}

\subsubsection*{Notes}
An object immersed in water experiences upward force equal to the weight of the water it displaces. The upthrust competes with the downward pull of gravity which diminishes the weight of the object. If the upthrust is greater than the object's weight, it will float. Otherwise, the object will sink. If the water density is greater than the average density of the object, the object will also float. If the water density is less than the average density of the object, the object will sink.  
