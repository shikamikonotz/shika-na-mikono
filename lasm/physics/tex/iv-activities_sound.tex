\section{Sound}

\subsection{Construction and Use of a Simple Sonometer}

\subsubsection*{Learning Objectives}
\begin{itemize}
\item{To construct and use a simple sonometer} 
\item{To explain the propogation of waves on a string} 
\end{itemize}

\subsubsection*{Background Information}
Sound waves can be produced when a string vibrates. The frequency at which the string vibrates depends on several things, like the length of the string and the material. A sonometer consists of a metal string which can vibrate between two supports. This is a standing wave and is driven by a tuning fork or other source of sound. If the natural frequency of the string (the frequency it will have because of its length and material) is the same as the tuning fork, the string will vibrate.  

\subsubsection*{Materials}
Soft wood board about 80 cm long, thin wire (steel works best) or string, nails, 2 small triangular pieces of wood(pegs) or two pencils, heavy stone

\subsubsection*{Preparation Procedure}
\begin{enumerate}
\item{Place the soft wood on a table.} 
\item{Fix a string/wire with a nail to one end of the soft wood.} 
\end{enumerate}

\subsubsection*{Activity Procedure}
\begin{enumerate}
\item{Hang the heavy mass of a stone to the free end of the string/wire so that the mass hangs below the edge of the table.} 
\item{Insert the two pegs/triangular pieces under the string/wire so as to raise the wire off the surface of the wood.} 
\item{Pluck the string/wire between the two pegs. If the wire does not make a clear note, add mass to the end.} 
\item{Vary the distance between the two pegs (increase and decrease) and observe the effect on the frequency of the wire/string.} 
\item{Vary the mass hanging on the end of the wire (increase and decrease) and observe the effect on the frequency of the wire/string.} 
\end{enumerate}

\subsubsection*{Results and Conclusion}
1. A higher tone is heard/produced if the distance between the two pegs is reduced.  
2. A higher tone is heard/produced if the mass is increased.  
The tone which is produced by the vibrating string/wire depends on the its vibrating length and the tension on the string/wire

\subsubsection*{Clean Up Procedure}
\begin{enumerate}
\item{Remove the mass from the string/wire.} 
\end{enumerate}

\subsubsection*{Discussion Questions}
\begin{enumerate}
\item{What do you hear in steps 4 and 5?}
\end{enumerate}

\subsubsection*{Notes}
The same activity can be used to show that the frequency of a vibrating tuning fork is invesly proportional to the length of a vibrating sting. It can also show that the frequency of a vibrating tuning fork is directly proportional to the tension in a vibrating string. Using different tuning forks and finding the corresponding length and tabulating the values, a graph of Frequency against Reciprocal length can be drawn and is a staight line passing through the origin. Also, wires of different diameters can be fixed on the plane wood. Collect the plane wood, nails from a nearby fundi and it cost about Tsh.  3, 000/=. Instead of heavy stones, dry sand packed in plastic bags can be used. Also, instead of two pegs you can use two pencils.  

\subsection{Wave Propogation in Solids}

\subsubsection*{Learning Objectives}
\begin{itemize}
\item{To observe the propogation of vibrations through a solid}
\item{To understand how sound is transmitted through a medium}
\end{itemize}

\subsubsection*{Background}
Sound is a pressure wave, which means it can travel through any medium so long as the molecules in that medium are free to vibrate.  In fact, this is every medium, though air and water transmit sound much more efficiently than solids.  However, solids can transmit waves, as in the case of a guitar, which uses vibrating strings to produce sound waves.
If a wave is produced on one end of a string, the string transmits the energy of that wave to the other end.

\subsubsection*{Materials}
spoon, string 1 m

\subsubsection*{Preparation Procedure}
\begin{enumerate}
\item{Tie the spoon into the middle of the length of string so that it will hang freely when you hold the string ends.}
\item{Have a student hold the string ends to his or her temples or the bone just under his or her ears as you strike the spoon with a pen or other object.}
\end{enumerate}

\subsubsection*{Results and Conclusion}
The student will hear a clear sound when the spoon is struck.  The vibrations of the spoon propagate up the string and into
the student's head. Bone, especially around the temples and outer ear, resonates readily in response to sound.

\subsubsection*{Cleanup Procedure}
\begin{enumerate}
\item{Untie the string and return all materials to their proper places.}
\end{enumerate}

\subsubsection*{Discussion Questions}
\begin{enumerate}
\item{What causes the sound to be loud when the string is held to your head?}
\item{Why does the bone in front of your ear tansmit vibrations more easily than other bones?}
\item{What is the purpose of the string in this activity?}
\end{enumerate}

\subsubsection*{Notes}
The bone in front of your ear is the most resonant bone in the body, so it is ideal for transmitting vibrations and hearing sound.  The vibrations from the spoon are transmitted easily through the string and your skull, which you hear as a sound.

\subsection{Sound Amplifer}

\subsubsection*{Learning Objectives}
\begin{enumerate}
\item{To understand the amplification of mechanical waves}
\item{To observe the amplification of sound in a hollow cavity}
\end{enumerate}

\subsubsection*{Background Information}
Everything can vibrate.  If a wave drives the vibration of another object or surrounding air, we say that the wave is amplified as its amplification has increases with the resonating body.  This principle is used when marimbas are played inside gourds.

\subsubsection*{Materials}
plastic water bottle, string or thread, match or small stick

\subsubsection*{Preparation Procedure}
\begin{enumerate}
\item{Make a small hole in the bottom of the bottle.}
\item{String one end of the thread through the hole.}
\item{Tie the end on the inside of the bottle to the match or small stick so that it cannot be pulled back through the hole.}
\item{Leave the length of string hanging out of the bottom of the bottle}
\end{enumerate}

\subsubsection*{Activity Procedure}
\begin{enumerate}
\item{Pull the string taught and have a student hold the top of the bottle.}
\item{Pluck the string.}
\item{Try plucking just the string and then the string and bottle together.}
\item{Try plucking the string with the cap on or off.}
\item{Observe the various effects of the sound.}
\end{enumerate}

\subsubsection*{Results and Conclusion}
When the string is plucked by itself, the sound it creates is very small.  However, when the string is attached to the bottle, the sound is louder.  The vibration of the string causes the bottle itself to vibrate.  Rather than hearing just the sound of the string vibrating, we hear the sound of the bottle, which produces noticeably greater amplitude.

\subsubsection*{Cleanup Procedure}
\begin{enumerate}
\item{Return all materials to their proper places.}
\end{enumerate}

\subsubsection*{Discussion Questions}
\begin{enumerate}
\item{What was the difference between the sound produced by the string and the sound produced by the string and bottle together?}
\item{What causes the sound you hear to be louder?}
\item{What was the difference in sound between using the cap and not using the cap?}
\end{enumerate}

\subsubsection*{Notes}
This effect can be difficult to detect if the bottle is small or if the frequency of the string is much higher or lower than that of the bottle.  Vary the length of the string until you get clear resonance.

\subsection{Determination of Resonance Frequency}

\subsubsection*{Learning Objectives}
\begin{itemize}
\item{To explain the concept of resonance as applied to sound} 
\end{itemize}

\subsubsection*{Background Information}
Every object has a natural frequency depending on its size, shape, material, etc.  If a wave drives the object at its natural frequency, the object itself will begin to vibrate along with the wave.  This effect is called resonance.

\subsubsection*{Materials}
Flourescent tube (tube light), thick rubber tubbing, two 1.5 litre plastic water bottles, super glue, wax, turning Fork, retort stand, bucket, water, long stick, knife, metre rule, rubber or cork, piece of cloth

\subsubsection*{Hazards and Safety}
\begin{itemize}
\item{Precautions should be taken when cutting the pipe as it will be sharp.}
\item{Do not touch the flourescent dust in the tube; it is poisonous.} 
\item{Use glue carefully, do not touch it with you bare fingers.} 
\end{itemize}

\subsubsection*{Preparation Procedure}
\begin{enumerate}
\item{Create a hollow tube from the florencent tube by cutting its rims off on both sides.} 
\item{Clean the tube with a piece of cloth attached to a long stick}
\item{Cut the bottom 5 cm off of one of the bottles (bottle 1) and cut the top 5 cm off the top the other bottle (bottle 2).} 
\item{Make a hole in the cap of both bottles.} 
\item{Attach one end of the pipe with the glue and wax to the inside top of bottle 2. Insert the rubber tubbing through the holes of both bottle caps.} 
\item{Hold the tube with retort stand together with a metre rule upright(vertically)}
\item{Raise bottle 1 vertically until you have created a U-shape.} 
\item{Pour water into bottle 1.} 
\end{enumerate}

\subsubsection*{Activity Procedure}
\begin{enumerate}
\item{Strike the turning fork with the soft material such as rubber}
\item{Place the turning fork at the top of the tube}
\item{rise and lower the water level in the tube by changing the vertical position of bottle 1}
\item{Repeat for the other two more different turning fork}
\item{For each turning fork note the fundamental note and overtone.} 
\end{enumerate}

\subsubsection*{Results and Conclusion}
Student should hear the tube resonating at two or more water levels. The lowest water level is the fundamental and each smaller water level are higher harmonics.  
Student should understand that: 
The length of the tube from the water to the top can be used to calculate speed of sound in air. 
Resonace frequency occurs when the natural frencency of the air column is eqaul to the forced frequecy from the tuning fork.  

\subsubsection*{Clean Up Procedure}
\begin{enumerate}
\item{Water from the pipe and reservior pour in the bucket}
\item{Keep the system on the shelves}
\end{enumerate}

\subsubsection*{Discussion Questions}
\begin{enumerate}
\item{Design another way to perform the same experiment.} 
\item{Think of material that can be used instead of florencent tube and flexible pipe}
\end{enumerate}

\subsubsection*{Notes}
The vibrating air column in the pipe produces a loud sound when the node of a waveform is at a closed end and the antinode at the openend. At this point the vibrating air is at resonance with the vibrating tuning fork is held at the open end. 
Teacher should also know that: 
This experience can be used to determine the velocity of sound in air. In absence of the listed materials using the tub in a bucket of water will also work.  

\subsection{Speed of Sound in Air}

\subsubsection*{Learning Objectives}
\begin{itemize}
\item{To understand the relationship between a wave's frequency, speed and wavelength} 
\item{To calculate the speed of sound in air using a resonance tube} 
\end{itemize}

\subsubsection*{Background Information}
A wave's speed or velocity is directly related to the wave's frequency and wavelength. If the wavelength and frequency can be determined, the speed can be calculated easily. We can use a resonance tube or sonometer to find the wavelength and frequency of a wave, therefore allowing us to calculate the speed of the sound wave in that medium.  

\subsubsection*{Materials}
Resonance tube (this can be made: see the activity about constructing a resonance tube in this book), tuning fork or wind instrument like a flute or recorder, water, metre rule

\subsubsection*{Hazards and Safety}
\begin{itemize}
\item{If you are using tuning forks, do not hit them on a table or any other hard object. Over time, this will damage the tuning forks and changes their frequencies until they are no longer useful. Instead, hold the middle of the handle and hit one of the fork's prongs on the sole of your shoe or any other hard rubber object.} 
\end{itemize}

\subsubsection*{Preparation Procedure}
\begin{enumerate}
\item{Set up the resonance tube with water.} 
\item{Place the metre rule next to the resonance tube so that the 0 cm mark is at the top of the resonance tube.} 
\end{enumerate}

\subsubsection*{Activity Procedure}
\begin{enumerate}
\item{Use a tuning fork or flute to make a single musical note.} 
\item{Place the fork or flute over the resonance tube and adjust the water level until the resonance can be heard in the tube. The note produced by the fork or flute will be heard at the top of the tube when the water is at a certain level.} 
\item{Record the distance between the water level and the top of the tube at the level when the water reaches a level where resonance can be heard.} 
\item{Also record the frequency of the note. If you are using tuning forks, the frequency is written on the handle. If you are using a flute, the frequency depends on the note you are playing. A table of musical notes and their frequencies can be found in any textbook. The easiest to use is A, which has a frequency of 440 Hz.} 
\item{Use your values of length (wavelength) and frequency to calculate the speed of sound in the tube.} 
\item{Repeat these steps for several notes and compare your values of speed of sound.} 
\end{enumerate}

\subsubsection*{Results and Conclusion}
It will be seen that the product of wavelength and frequency will be almost the same. This is because the speed of sound in air is constant (depending on humidity and density). Therefore, the product of wavelength and frequency must be constant. As frequency increases, the wavelength (the length of the tube above the water level) decreases.  

\subsubsection*{Clean Up Procedure}
\begin{enumerate}
\item{Empty the water from the resonance tube.} 
\item{Return all materials to their proper places.} 
\end{enumerate}

\subsubsection*{Discussion Questions}
\begin{enumerate}
\item{What was your average value for the speed of sound in air?}
\item{As the frequency of the musical note increased, did its wavelength increase or decrease?}
\item{If the speed of a wave remains constant, what is the relationship between wavelength and frequency of a wave?}
\end{enumerate}

\subsubsection*{Notes}
There is room for error in this experiment because you are trying to measure length as the water level is moving. Also, the flute or other instrument you are using to create a note may not be perfectly in tune and so may have a slightly different frequency. Do several experiments in order to find a consistent value for the wavelength.
