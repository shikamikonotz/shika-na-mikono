\section{Newton's Laws and Forces}

\subsection{Inertia and Newton's First Law of Motion}

\subsubsection*{Learning Objectives}
\begin{itemize}
\item{To understand the concept of inertia} 
\item{To apply Newton's First Law (the Law of Inertia)} 
\item{To explain the effects of inertia on a moving or stationary body} 
\end{itemize}

\subsubsection*{Background Information}
The inside of a fresh egg is liquid while the inside of a boiled egg is solid. Therefore, if you change the motion of the shell of a fresh egg, the liquid inside will resist the change and continue with whatever motion it had. If you change the motion of a boiled egg shell, the inside of the egg will follow the same motion as the shell.  

\subsubsection*{Materials}
1 fresh egg and 1 boiled egg

\subsubsection*{Activity Procedure}
\begin{enumerate}
\item{Place both eggs on the table and note that it is impossible to tell which egg is fresh and which egg is boiled.} 
\item{Spin the first egg.} 
\item{While the egg is spinning, stop it briefly with your hand and then release the egg.  Record any observations.} 
\item{Spin the second egg.} 
\item{While the egg is spinning, stop it briefly with your hand and then release the egg.  Record any observations.} 
\end{enumerate}

\subsubsection*{Results and Conclusion}
The fresh egg, which is liquid inside, will continue spinning even after its rotation is stopped briefly by your hand. The boiled egg, which is solid inside, will stop spinning after its rotation is stopped briefly by your hand.  
The fresh egg continues spinning because the liquid inside continues to spin and causes the shell to move with it. However, the boiled egg stops spinning because the solid inside has stopped moving and thus will remain stationary.  

\subsubsection*{Discussion Questions}
\begin{enumerate}
\item{Which egg is fresh and which egg is boiled?}
\item{Why does the boiled egg stop completely when your hand releases it while the fresh egg continues spinning?}
\item{Explain the motion of the eggs in terms of inertia.} 
\end{enumerate}

\subsubsection*{Notes}
Newton's first law, also called the Law of Inertia, states that "an object in motion will continue in that motion, and an object at rest will remain at rest unless acted upon by an external force". This simply means that an object will continue doing what it is doing and will resist any changes.  
For this reason, the spinning inside of an egg will coninue to spin even when the outer shell is stopped. When you release the egg, the shell is pulled along with the liquid. However, the solid egg, when stopped, has no more inertia of motion and will remain at rest until you spin it again.  

\subsection{Conservation of Linear Momentum}

\subsubsection*{Learning Objectives}
\begin{itemize}
\item{To demonstrate the principle of conservation of linear momentum}
\end{itemize}

\subsubsection*{Background Information}
Everything has momentum which depends on its mass and velocity.  The momentum of an individual body can change as its velocity or mass changes.  Hoever, if two objects collide, the total momentum of the objects is conserved.  This means that the total momentum of the objects before the collision is equal to their total momentum after the collision.

\subsubsection*{Materials}
Toy car with motor, plane surface or smooth table, metre rule or tape measurer, beam balance*, different sized stones, and stop watch

\subsubsection*{Preparation Procedure}
\begin{enumerate}
\item{Collect all necessary materials.}
\item{Measure the masses of different stones on the beam balance.}
\item{Measure the mass of the toy car.}
\item{Measure a distance of 2 m on the plane surface or table.}
\item{Make a mark at 0 m and place an obstacle at 2 m.}
\end{enumerate}

\subsubsection*{Activity Procedure}
\begin{enumerate}
\item{Place the toy car at the 0 m mark on the table.}
\item{Release the car and start your stop watch.}
\item{Record the time it takes for the car to move from the 0 m mark to the obstacle at the 2 m mark.}
\item{Use this time and distance to calculate the average velocity of the car.}
\item{Place a stone on top of the toy car (use tape or string if necessary in order to secure it).}
\item{Measure the new mass of the car with the stone on top.}
\item{Start the car and release it on the table at the 0 m mark.}
\item{Again, measure the time it takes for the car to reach the obstacle at the 2 m mark.}
\item{Calculate the average velocity of the car and stone.}
\item{Repeat these steps for various stones, measuring the masses and average velocities for each case.}
\item{For each case, calculate the momentum of the car and stone.}
\item{Record your results in a table.  Fill in values for mass, time, velocity and momentum.}
\item{Compare the values for momentum.}
\end{enumerate}

\subsubsection*{Results and Conclusion}
The momentum for each experiment is almost the same.  As the mass increases, the velocity decreases.  However, the product of the two (momentum = velocity x mass) remains the same.  However, the momentum decreases slightly with increased mass because friction on the axles of the car is also increased.
When two bodies, one heavy and one light, are acted upon by the same force for the same amount of time, the lighter object's velocity increases more than that of the heavy object.  However, the momentum that each gains is the same.

\subsubsection*{Clean Up Procedure}
\begin{enumerate}
\item{Return all materials to their proper places.}
\end{enumerate}

\subsubsection*{Discussion Questions}
\begin{enumerate}
\item{What factors affect the momentum of the car?}
\item{When the mass of the car is increased by adding stones, what happens to the average velocity?}
\item{What do you observe when comparing the values for momentum?}
\end{enumerate}

\subsubsection*{Notes}
Conservation of momentum is only true in a frictionless environment.  However, the effects can be seen clearly even if friction is present.  The toy car has friction between its wheels and axles, so adding mass to the car will increase the effect of friction.  However, it will still be seen that the momentum is relatively equal for each mass.

\subsection{Bottle Rocket}

\subsubsection*{Learning Objectives}
\begin{itemize}
\item{To observe the effect of Newton's Third Law of Motion}
\end{itemize}

\subsubsection*{Background Information}
Newton's Third Law tells us that, for every action, there is an equal and opposite reaction.  This means that if you apply a force to something, it applies an equal force back on you.  Rockets make use of this principle by ejecting gas at high speeds out of one end so that they are forced in the opposite direction.

\subsubsection*{Materials}
empty 500 mL water bottle, nail, rubber stopper, straight pin, bicycle pump, needle attachment for pump (the type used to fill a football), tape, old pen, rigid straight wire (approx 1 meter), water

\subsubsection*{Preparation Procedure}
\begin{enumerate}
\item{Make a small round hole (between 0.  5 and 1.  0 cm in diameter)in the lid of the water bottle by heating a nail and using it to put a hole in the lid.}
\item{Cut a round piece of the rubber stopper so that it can be used to stop this hole. The stopper should form a good seal in this hole, but it should be possible to push the stopper through the hole by exerting some force on it.}
\item{Pierce the stopper with a straight pin so that you can pass the needle attachment for a bicycle pump through the stopper.}
\item{Push the stopper into the hole inside of the lid.}
\item{Insert the needle attachment through the stopper so that you can increase the pressure inside of the bottle.}@
\item{Disassemble a pen and take the plastic case.}
\item{Cut the case in half so that you have two hollow cylindrical pieces approximately 3cm long each.}
\item{Attach them to the side of the bottle using adhesive tape. They should be in a straight line with each other.}
\end{enumerate}

\subsubsection*{Activity Procedure}
\begin{enumerate}
\item{Insert the rigid straight wire into the ground outside.}
\item{Fill approximately half the bottle with water.}
\item{Push the stopper into the inside of the lid.}
\item{Put the needle attachment through the stopper.}
\item{Put the lid on the water bottle and tighten it so that air cannnot escape.}
\item{Pass the rigid wire through the pen cylinders and lower the bottle to the ground.}\item{Pump the bicycle pump until the stopped is pushed out completely.}
\end{enumerate}

\subsubsection*{Hazards and Safety}
\begin{itemize}
\item{This is a rocket and will take off quickly and travel far.  Be sure that no one is standing in the way of the rocket, and launch it in a large, open space where no one and nothing can be hit by it.}
\end{itemize}

\subsubsection*{Results and Conclusion}
Once the pressure in the bottle becomes great enough, the stopper will be forced out of the bottle, and the rocket will fly into the air. It should be possible to reach a height of 10 meters or more.  When the stopper leaves the bottle, pressurized air forces water out of the bottom of the bottle at a high speed. Just as with the matchstick rocket, this results in a reaction force forwards on the rocket.
We can also consider this from the perspective of conservation of momentum.

\subsubsection*{Cleanup Procedure}
\begin{enumerate}
\item{Clean up any water and retrieve the rocket.}
\item{Return materials to their proper places.}
\end{enumerate}

\subsubsection*{Discussion Questions}
\begin{enumerate}
\item{What causes the rocket to launch?}
\item{Explain the two opposing forces present in this experiment.}
\end{enumerate}

\subsubsection*{Notes}
All rockets use this principle; that rapidly expanding gases in one direction causes motion in the opposite direction.  This combines Newton's third law and conservation of momentum.
This activity takes practice.  Test this several times before doing it with students.  

\subsection{Matchstick Rocket}

\subsubsection*{Learning Objectives}
\begin{itemize}
\item{To explain the mode of action of a rocket} 
\item{To apply Newton's Third Law to the motion of a rocket} 
\item{To understand the application of Newton's Third Law to propulsion} 
\end{itemize}

\subsubsection*{Background Information}
The motion of a rocket is due to the simple principle of reaction.  Newton's third law explains that a force in one direction will be opposed by a force in the opposity direction.  In other words, if an onject pushes backwards against an obstacle, the obstacle will push forward on the object.  When a match burns, the gases that are produced are very hot and expand rapidly. In order for the match to continue burning, the match pushes the gases backwards, and the gases need to escape.

\subsubsection*{Materials}
matches, aluminium foil, pin or syringe needle

\subsubsection*{Hazards and Safety}
\begin{itemize}
\item{While they are not very dangerous, matches can ignite anything that is paper or cloth or flammable quite easily. For this reason, make sure that no one and nothing is in the path of the rockets or anywhere near the launch site.} 
\item{When disposing of the trash, be sure that all matches have stopped burning and are not hot so that you do not ignite any other trash.} 
\item{Aluminuium foil does not decompose, so all pieces should be collected and disposed of properly.} 
\item{When the rocket ignites, some foil may be expelled, so no one should be very close to the rocket when it ignites. When igniting it yourself, keep your face away from the rocket.} 
\end{itemize}

\subsubsection*{Preparation Procedure}
\begin{enumerate}
\item{Collect all materials.} 
\item{If you are using a syringe needle instead of a pin, break the needle near the plastic base so that your needle is only the metal part.} 
\end{enumerate}

\subsubsection*{Activity Procedure}
\begin{enumerate}
\item{Cut or rip a small piece of foil, about 2 cm x 3 cm. Make sure it is flat and does not contain any holes.} 
\item{Hold the pin next to a match so that the tip of the pin touches the head of the match.} 
\item{Hold the pin and match tightly together and wrap the foil around the head of the match (with the pin) so that about 1 cm of foil covers the match and pin, and about 1 cm extends beyond the tip of the match and pin. Wrap the foil tightly so that no openings can be seen around the shaft of the match and pin.} 
\item{Fold the extra foil down over the match head tightly so that there are no openings.} 
\item{Remove the pin by sliding it out of the bottom of the foil, leaving a thin tunnel.} 
\item{Support the match rocket at a 45-degree angle on a stone or other object. Make sure that no one is standing in front of the rocket.} 
\item{Light another match and hold it under the foil of the rocket. It may take a few seconds to work.} 
\item{Repeat all steps until you have a good rocket.} 
\end{enumerate}

\subsubsection*{Results and Conclusion}
The matchstick rocket moves forward quickly when the matchhead ignites inside the foil.  The match moves forward because the gases are moving backward; a force in one direction must be balanced by an equal force in the opposite direction.  

\subsubsection*{Clean Up Procedure}
\begin{enumerate}
\item{Clean up all of the pieces of foil and match and dispose of them in the trash. Be sure that no matches are still burning.} 
\item{Return the other supplies to their proper places.} 
\end{enumerate}

\subsubsection*{Discussion Questions}
\begin{enumerate}
\item{What causes the rocket to move forward?}
\item{Where do the gases from combustion of the match go when the rocket ignites?}
\item{What would happen if the exhaust hole left by the pin was facing another direction?}
\item{What will happen if you increase the amount of gas escaping from the match?}
\item{What will happen if you increase the weight of the match?}
\end{enumerate}

\subsubsection*{Notes}
As a match ignites, the gases quickly expand. By removing the pin from the foil over the match head, you leave a small path for the gases to expand: they cannot go any other direction but backward. Newton's Third Law tells us that "For every action (or force) there is an equal and opposite reaction.  " The action in this case is the rapid movement of gases backward. The reaction produced is the movement of the matchstick forward. This is the mode of action for all rockets.  

\subsection{Verification of Newton's First Law of Motion}

\subsubsection*{Learning Objectives}
\begin{itemize}
\item{To explain the concept of inertia} 
\item{To Verify Newton's first law of motion} 
\end{itemize}

\subsubsection*{Background Information}
An object will tend to resist changes to its motion. This is called inertia and is explained by Newton's first law of motion. This law states that "an object at rest will tend to stay at rest unless acted upon by an external force. An object in motion well continue in that motion unless acted upon by an external force.  " In short, this means that an object will continue doing what it is doing.  

\subsubsection*{Materials}
Empty soda bottle, small manila sheet OR card, knife, coin

\subsubsection*{Hazards and Safety}
\begin{itemize}
\item{Take care to use a knife and to flick the card quickly and to the side.} 
\end{itemize}

\subsubsection*{Preparation Procedure}
\begin{enumerate}
\item{Find an empty and dry soda bottle.} 
\item{Cut a small piece of card from the manila sheet.} 
\item{Find a coin.} 
\end{enumerate}

\subsubsection*{Activity Procedure}
\begin{enumerate}
\item{Place the bottle on top of the table.} 
\item{Cover the mouth of the bottle with the card.} 
\item{Put a coin on top of the card on the rim of the bottle.} 
\item{Flick or quickly pull the piece of manila sheet horizonatally off of the bottle.} 
\end{enumerate}

\subsubsection*{Results and Conclusion}
As you flick the piece of card quickly the card flies away and the coin remains in the same position (on top of the bottle rim). The coin remains at the same point due to its tendency to maintain its position. This property of resisting change to motion is called inertia. It is also explained by Newton's first Law of Motion.  

\subsubsection*{Clean Up Procedure}
\begin{enumerate}
\item{Return the materials to their proper places.} 
\end{enumerate}

\subsubsection*{Discussion Questions}
\begin{enumerate}
\item{Discuss your daily experience with the concept of inertia.} 
\item{Design another way to demonstrate the concept of inertia.} 
\item{Tell why Newton's first law is known as the "Law of Inertia.  "}
\end{enumerate}

\subsubsection*{Notes}
The tendency of a body to remain at the same point when it is at rest or to maintain its direction when it is in motion is known as inertia. The concept of inertia explains the reason why passengers move forward when the driver applies breaks, or fall back when the car starts moving suddenly.
