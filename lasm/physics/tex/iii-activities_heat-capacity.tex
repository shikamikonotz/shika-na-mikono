\section{Heat Capacity and Latent Heat}

\subsection{Latent Heat of Fusion}

\subsubsection*{Learning Objectives}
\begin{itemize}
\item{To observe the process of fusion} 
\item{To explain the difference between heat capacity and latent heat} 
\item{To observe the constant temperature of latent heat} 
\end{itemize}

\subsubsection*{Background Information}
As a substance is heated, there are two types off heat involved. Heat capacity is the heat needed to raise the temperature of the substance; latent heat is the heat needed to change the substance from one state to another.  
Heat capacity is the heat that we normally see: you can measure it with a thermometer as it raises the temperature of a body. Latent heat, however, is "hidden.  " This means that latent heat does not raise the temperature of a body, so it cannot be measured with a thermometer.  
When a substance is heated, its temperature increases as it gains heat as per its heat capacity. However, when it changes state from solid to liquid or liquid to gas, its temperature remains constant as it is absorbing latent heat.  

\subsubsection*{Materials}
heat source, water, small cooking pot, thermometer.

\subsubsection*{Preparation Procedure}
\begin{enumerate}
\item{The thermometer may be a real thermometer if you want to measure the temperature at which latent heat is absorbed. However, you can use any liquid in a capillary tube to see the expansion of the liquid. When the latent heat is used and the water is changing state, the liquid in the tube will stop expanding.}
\end{enumerate}

\subsubsection*{Activity Procedure}
\begin{enumerate}
\item{Fill the pot about half-way with water.} 
\item{Place the thermometer in the water.} 
\item{Turn on the heat and place the pot over the heat.} 
\item{Observe the rise in temperature as the water is heated. Record the temperature every 10 seconds.} 
\item{Continue recording the temperature every 10 seconds after the water begins to boil.} 
\item{Plot a graph of temperature (vertical axis) against time (horizontal axis).} 
\end{enumerate}

\subsubsection*{Results and Conclusion}
The temperature will be seen to increase steadily as the water is heated. When the water begins to boil, the temperature stops increasing and remains constant while the water vapoourizes. The graph will show a steadily increasing temperature until it reaches the boiling point on the vertical axis. This value should be about 100 degrees C. At this point, the temperature will level off as time continues to increase.  

\subsubsection*{Clean Up Procedure}
\begin{enumerate}
\item{Remove the thermometer from the water and pour it out.} 
\item{Turn off the heat and return all materials to their proper places when they are cooled.} 
\end{enumerate}

\subsubsection*{Discussion Questions}
\begin{enumerate}
\item{What happens to the temperature when the water begins to boil?}
\item{What happens to the temperature as the water is heated but before it boils?}
\item{What heat is involved in heating the water?}
\item{What heat is involved in changing the water from a liquid to vapour?}
\end{enumerate}

\subsubsection*{Notes}
It is important to periodically measure the temperature and time as the water heats.  This will allow you to see clearly the boiling point when the laten heat of water becomes a factor.
