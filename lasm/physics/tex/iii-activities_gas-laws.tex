\section{Gas Laws}

\subsection{Boyle's Law}

\subsubsection*{Learning Objectives}
\begin{itemize}
\item{To observe the relationship between volume and pressure when temperature is kept constant} 
\item{To explain the relationship between pressure and volume} 
\item{To explain Boyle's Law} 
\end{itemize}

\subsubsection*{Background Information}
Gases can be measured by three quantities: pressure, volume and temperature. For this reason, we have laws to relate them to each other. When one quantity is held constant, the other two prove to be either directly proportional or inversely proportional.  

\subsubsection*{Materials}
Syringes without the needle for each student

\subsubsection*{Preparation Procedure}
\begin{enumerate}
\item{Collect the syringes and remove the needles.} 
\end{enumerate}

\subsubsection*{Activity Procedure}
\begin{enumerate}
\item{Pull the syringe plunger back as far as it will go without removing it.} 
\item{Place your thumb over the open end of the syringe.} 
\item{Push the plunger in as hard and as far as you can.} 
\item{Observe any effects.} 
\item{Remove your thumb from the syringe and push the plunger in as far as it will go.} 
\item{Replace your thumb over the opening of the syringe.} 
\item{Pull the plunger out as hard and as far as you can.} 
\item{Observe any effects.} 
\item{Have students try this themselves.} 
\end{enumerate}

\subsubsection*{Results and Conclusion}
It will be seen that as you push the plunger in, there is a strong force pushing both the plunger and their thumb out.  This is the increased pressure of the air inside the syringe.  As you pull the plunger out, there is a strong force pulling both the plunger and your thumb into the syringe. This is the decreased pressure of the air inside the syringe.  As volume is decreased, pressure is increased; as volume is increased, pressure is decreased.  From this we can say that pressure and volume are inversely proportional.  

\subsubsection*{Clean Up Procedure}
\begin{enumerate}
\item{Collect all syringes and return them to their proper place.} 
\end{enumerate}

\subsubsection*{Discussion Questions}
\begin{enumerate}
\item{What quantity is being held constant in this experiment?}
\item{What quantities are changing in this experiment?}
\item{What do you feel on your thumb when you are trying to pull the plunger out? What do you feel when you are trying to push it in?}
\item{Describe the relationship between pressure and volume in this experiment.} 
\item{Based on this experiment, state Boyle's Law.} 
\end{enumerate}

\subsubsection*{Notes}
Pressure varies inversely with volume. As you push the plunger in, you are decreasing the volume and therefore increasing the pressure, which you feel as an outward force.  
As you pull the plunger out, you are increasing the volume and therefore decreasing the pressure, which you feel as an inward force.  


\subsection{Charles's Law}

\subsubsection*{Learning Objectives}
\begin{itemize}
\item{To observe the relationship between volume and temperature when pressure is constant} 
\item{To explain the meaning of Charles' Law in terms of pressure, volume and temperature} 
\end{itemize}

\subsubsection*{Background Information}
Gases can be measured by three quantities: pressure, volume and temperature. For this reason, we have laws to relate them to each other. When one quantity is held constant, the other two prove to be either directly proportional or inversely proportional.  

\subsubsection*{Materials}
1.  5 Litre water bottle with cap, heat source, small cooking pot, water

\subsubsection*{Hazards and Safety}
\begin{itemize}
\item{Be careful when pouring the water from the pot to the bottle. Use a funnel if necessary.} 
\end{itemize}

\subsubsection*{Preparation Procedure}
\begin{enumerate}
\item{Collect all materials.} 
\end{enumerate}

\subsubsection*{Activity Procedure}
\begin{enumerate}
\item{Light the heat source.} 
\item{Fill the small pot with water and place it over the heat source.} 
\item{Allow the water to boil.} 
\item{When the water is boiling, pour it into the water bottle.} 
\item{Close the cap on the bottle and shake the water.} 
\item{Pour the water out of the bottle and close the cap again.} 
\item{Observe any changes to the bottle.} 
\end{enumerate}

\subsubsection*{Results and Conclusion}
After pouring the boiled water into the bottle and closing the cap, the volume of air in the bottle increases.  This is because the air is gaining heat, and therefore kinetic energy, from the water and therefore increasing in volume.  Pouring the water out removes the air's source of heat.  When you replace the cap on the bottle, you return the bottle to a state of constant pressure.  As the air in the bottle cools, the temperature goes down.  Temperature and volume are directly related, so the volume of the bottle decreases.  We see this as the bottle being crushed.

\subsubsection*{Clean Up Procedure}
\begin{enumerate}
\item{Turn off the heat source.} 
\item{Dispose of the water and return all materials to their proper places.} 
\end{enumerate}

\subsubsection*{Discussion Questions}
\begin{enumerate}
\item{What quantity of a gas is constant during this experiment?}
\item{Which two quantities are changing?}
\item{What causes the bottle to be crushed?}
\item{Based on this experiment, what is Charles's Law?}
\end{enumerate}

\subsubsection*{Notes}
Charles's Law states that, when pressure is constant, temperature varies directly with volume. This means that, in an air-tight system, the volume will increase when something is heated or decrease when something is cooled. This is the principle behind thermal expansion and can be seen easily in the thermal expansion demonstrations.  
In this demonstration boiling water is used initially to increases the temperature of the air in the bottle. It is removed and the bottle is sealed, forcing the pressure to remain constant. As the air inside the bottle cools, it decreases the volume, causing the bottle to be crushed from the inside.  EQUATION  T : V when P is constant
