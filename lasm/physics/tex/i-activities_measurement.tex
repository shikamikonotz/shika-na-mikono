\section{Measurement}	

\subsection{Construction of a Metre Rule}

\subsubsection*{Learning Objective}
\begin{itemize}
\item{To construct and use a metre rule.} 
\end{itemize}

\subsubsection*{Background Information}
Length is an interval between two points.  It is usually measured in metric units like the metre (m), millimetre (mm), centimetre (cm), kilometre (km).

\subsubsection*{Materials}
Wooden board, pen/pencil, a handsaw, ruler or tape measure

\subsubsection*{Preparation Procedure}
\begin{enumerate}
\item{Use the saw to cut a piece of wood 100 cm x 3.5 cm x 0.5 cm.} 
\item{Use a ruler to mark a scale in cm on the wood.} 
\end{enumerate}

\subsubsection*{Activity Procedure}
\begin{enumerate}
\item{Use the new metre ruler to measure length of different objects.} 
\end{enumerate}

\subsubsection*{Discussion Questions}
\begin{enumerate}
\item{What other objects can be arranged in order to measure distance?}
\end{enumerate}

\subsubsection*{Notes}
Instead of a wooden block, string can be used by making knots at a definite intervals.  Units of length were chosen by scientists; we can create any unit to measure length.


\subsection{Construction of a Measuring Cylinder}

\subsubsection*{Learning Objective}
\begin{itemize}
\item{To construct and use a measuring cylinder.} 
\end{itemize}

\subsubsection*{Background Information}
Volume of a liquid can be measured easily by using a measuring cylinder.  Volume is measured in millilitres (mL).

\subsubsection*{Materials}
Plastic bottles of different sizes, syringes (30-50 mL), marker pen, ruler, bucket full of water

\subsubsection*{Preparation Procedure}
\begin{enumerate}
\item{Using the syringe, take a known volume of water from the bucket.} 
\item{Transfer the known volume of water in the syringe to the empty plastic bottle.} 
\item{Using the marker pen, mark the level of water in the plastic bottle with the volume from the syringe.}
\item{Repeat this step for a range of volumes, marking each volume on the side of the bottle.}
\item{Use a ruler to complete the scale.} 
\end{enumerate}

\subsubsection*{Activity Procedure}
\begin{enumerate}
\item{Use the constructed measuring cylinder to measure volumes of different liquids}
\end{enumerate}

\subsubsection*{Clean Up Procedure}
\begin{enumerate}
\item{Dispose of all waste and return materials to their proper places.}
\end{enumerate}

\subsubsection*{Discussion Question}
\begin{enumerate}
\item{Explain how to use a measuring cylinder.}
\item{Why can't we use a ruler to measure the volume of a liquid?}
\end{enumerate}

\subsubsection*{Notes}
A measuring cylinder cannot be used to measure the volume of a solid by itself, though if the solid is immersed in a liquid, the volume can be determined with a measuring cylinder.  The volume of granular and porous materials can also be measured if they are immersed in a liquid in a measuring cylinder.

\subsection{Construction of Beam Balance}

\subsubsection*{Learning Objective}
\begin{itemize}
\item{To construct and use a beam balance.} 
\end{itemize}

\subsubsection*{Background Information}
Mass is a fundamental quantity and is measured by using a spring balance or beam balance.  Mass is usually measured in grams (g) and kilograms (kg).

\subsubsection*{Materials}
wooden board 30 cm x 2 cm, string/wire, ruler, pencil/pen, heat source, 2 large plastic bottles, nails, heat source*

\subsubsection*{Preparation Procedure}
\begin{enumerate}
\item{Cut a piece of wood block to 30 cm x 2 cm.} 
\item{Using the pen, find the balancing point.} 
\item{Make a hole at the balance point using a hot nail.} 
\item{Mark 5cm spaces on each side of the hole using a ruler.} 
\item{Cut the bottom out of the 2 plastic bottles about 3 cm-4 cm from the bottom.} 
\item{Make 3 round holes in the bottom of a plastic bottles at equal intervals. These will be used as scale pans.} 
\item{Tie pieces of thread/wire about 15 cm length into the holes of the bottom of a plastic bottle.} 
\item{Join the upper ends of the thread/wires together.} 
\item{Suspend the wooden block by using a piece of string/wire tied through the centre hole.} 
\end{enumerate}

\subsubsection*{Activity Procedure}
\begin{enumerate}
\item{Use the beam balance to measure the masses of different objects in the classroom}
\end{enumerate}

\subsubsection*{Clean Up Procedure}
\begin{enumerate}
\item{Return all materials to their proper places}
\end{enumerate}

\subsubsection*{Discussion Questions}
\begin{enumerate}
\item{How can you improve the beam balance?}
\end{enumerate}

\subsubsection*{Notes}
There are different kinds of balances, such as a digital balance, double beam balance, single beam balance and triple beam balance.  Each can be used to measure mass according to its sensitivity.

\subsection{Measurement Errors}

\subsubsection*{Learning Objectives}
\begin{itemize}
\item{To understand the meaning of experimental error}
\item{To understand the importance of accurate measurement and sample size}
\end{itemize}

\subsubsection*{Background Information}
Measurement is one of the most important aspects of science.  However, it is impossible to make a perfect measurement; we simply measure as well as we can because of our own errors and the tools that we use.  To improve our accuracy, we take many measurements and compare them to get an average result.  However, it is important to understand the source of our errors, how to account for them and how to reduce them.
Do this activity with a group of students to show that measurement is not the same from one person to another, or one instance to another.  Different people measure differently, and even one person will measure differently from one moment to another.

\subsubsection*{Materials}
Metre rules, stopwatches, other tools for measuring, materials to measure

\subsubsection*{Preparation Procedure}
\begin{enumerate}
\item{Collect different tools used for measuring, like metre rules or rulers, stopwatches, syringes, etc.}
\end{enumerate}

\subsubsection*{Activity Procedure}
\begin{enumerate}
\item{Draw a line on the board or floor.}
\item{Have several students measure the line and secretly record their results.}
\item{Collect the results from the students and record them on the board.  Observe any differences.}
\item{Distribute stopwatches to several students.}
\item{Clap twice; the students should measure the time between claps and record their results secretly.}
\item{Collect the results from the students and record them on the board.  Observe any differences.}
\item{Have several students make other similar measurements, keeping their results secret until you record them on the board.}
\item{Have students decide what the best result is for each of the collected measurements.}
\end{enumerate}

\subsubsection*{Results and Conclusion}
The measurements will be different from person to person.  Lengths, times, volumes, etc. will all vary for the same quantity.  This is because each person measures slightly differently; this is ok as long as each person is trying their best to be accurate.  The best result is the average of the results, combining the accuracy of all people in the final result.

\subsubsection*{Discussion Questions}
\begin{enumerate}
\item{Why are the measurements different from student to student?}
\item{How accurate are the tools that we use?}
\end{enumerate}

\subsubsection*{Notes}
Be sure that students understand that error is not bad.  Many science students feel that they need to make their answers better, even if it means changing their data.  Every measurement ever made has an error.  Some tools are better at measuring than others; we simply use what we have and measure as accurately as we can.  However, error needs to be understood and taken into account when doing experiments.