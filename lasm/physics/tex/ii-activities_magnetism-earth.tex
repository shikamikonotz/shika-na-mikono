\section{Earth's Magnetic Field}

\subsection{Creating a Simple Compass}

\subsubsection*{Learning Objectives}
\begin{itemize}
\item{To construct a compass and understand its mode of action}
\item{To observe the presence of Earth's magnetic field} 
\end{itemize}

\subsubsection*{Background Information}
The earth is a magnet.  Its magnetic field lines extend from its North pole (near the geographic South pole) to its South pole (near the geographic North pole).  However, we cannot see these lines as they are simply lines of force.
Any magnet feels the force of the earth's magnetic field and tries to turn to face North.  If a light-weight magnet is allowed to rotate freely, it will turn, thus showing the direction of North and South.

\subsubsection*{Materials}
Needle or pin, magnet, small plastic lid, water.

\subsubsection*{Preparation Procedure}
\begin{enumerate}
\item{collect the needle or pin.  If the it has a heave end, break it so that it is uniform.} 
\item{If needed get the bar magnet from a broken radio or speaker from the radio repair shop.}
\item{Clean the small can lid.} 
\end{enumerate}

\subsubsection*{Activity Procedure}
\begin{enumerate}
\item{Rub one side of the needle or pin on a bar magnet in one direction several times, do not scratch it.} 
\item{Pour some water into the can cap.} 
\item{Stitch the magnetized pin or needle into the piece of paper and place them on the surface of the water, let it rest. Observe it.} 
\item{give a slight push to the piece of paper so that it rotates slowly.  observe it.} 
\end{enumerate}

\subsubsection*{Results and Conclusion}
A magnetized pin or needle always comes to rest in the North-South direction.  This implies that the needle is pointing in the direction of earth's magnetic field towards the geographical north pole and magnetic south pole.  

\subsubsection*{Clean Up Procedure}
\begin{enumerate}
\item{Remove the stitched paper from the cap and pour water from the cap.} 
\end{enumerate}

\subsubsection*{Discussion Questions}
\begin{enumerate}
\item{Which direction does the magnetized pin always point?}
\end{enumerate}

\subsubsection*{Notes}
When magnetizing the pin or needle make sure you rub it only in one direction. Do not rub to and fro.  The geographical north and south poles are earth magnetic south and north poles respectively.  

\subsection{Magnetic Dip Gauge}

\subsubsection*{Learning Objectives}
\begin{itemize}
\item{To observe the presence of magnetic dip}
\item{To measure magnetic dip}
\end{itemize}

\subsubsection*{Background}
The earth's magnetic field extends from near the geographic South pole to the near geographic North pole.  However, its lines of force pass through the surface of the earth because the lines are not perfect circles around the earth.  Where the field passes through the surface of the earth, it has a certain angle which we call the Magnetic Dip, or the angle of the field lines relative to the earth's surface.

\subsubsection*{Materials}
magnet, sewing needle, cork or foam, two pins, paper, pen, cardboard or metal strip

\subsubsection*{Preparation Procedure}
\begin{enumerate}
\item{Push the two pins into the ends of the cork to create an axle.}
\item{Push the sewing needle through the cork perpendicular to the axle pins so that the needle rolls end-over-end when you roll the cork/pins between your fingers.}
\item{Adjust the needle so that it rests horizontally when the cork
is free to pivot (equilibrium).}
\item{Use the magnet to magnetize the needle without changing its position in the cork.  You can do this by stroking the needle in one direction on the magnet.}
\item{Bend the metal or cardboard strip into a U-shape to create a stand for the cork and pins.}
\item{Rest the pins on each vertical side of the U-stand so that the needle is free to rotate vertically.}
\item{Cut out a semicircular piece of paper and label the angles 0 to 90 degrees on it; tape or glue this to the stand.}
\end{enumerate}

\subsubsection*{Activity Procedure}
\begin{enumerate}
\item{Set up the magnetic dip gauge so that the needle is free to rotate vertically.}
\item{Measure the angle of the needle relative to the ground.}
\end{enumerate}

\subsubsection*{Results and Conclusion}
Before magnetizing the needle, it will be able to balance horizontally in equilibrium.  However, when the needle is magnetized, it will dip down to show the direction of the earth's field.  Like a compass, the needle naturally moves to show the direction of the earth's magnetic field.

\subsubsection*{Cleanup Procedure}
\begin{enumerate}
\item{Return all materials to their proper places and put the magnet in a safe place.}
\end{enumerate}

\subsubsection*{Discussion Questions}
\begin{enumerate}
\item{What is the direction of the needle?}
\item{Why does the needle not point horizontally, as it did before it was magnetized?}
\item{What is the angle between the needle and the ground?}
\end{enumerate}

\subsubsection*{Notes}
The magnetic dip gauge works only when it is facing North and South.  If it is facing East/West, the magnetic field will be moving perpendicular to the poles of the gauge, so it will not be able to show the correct direction.  Also, you may need to turn the gauge around if it is not showing the dip; if the needle is magnetized opposite to the direction of the earth's field, it will fail to show the correct direction.
